\section{Techniques to create Polyglot Files}
The techniques that I've used to created the polyglot files were inspired from parts of the presentation
given by Ange Albertini where he introduced 4 types of polyglot file creation namely, Cavities, Concatanation (or stacking), Parasites and
Zippers\cite{Ange-Albertini}.

\subsection{Concatanation, Stacks}
Concatanation is the simplest form of creating a polyglot, it's simply appending the the second file at the end of the first one.
Any offsets or pointers or references in the contents of the second file must be adjusted to new positions, this first file is left untouched.
The requirment for this is that the second file doesn't enforce $Magic$ $Number$ at offset 0 of its contents. As we have talked that file format specifications
are not perfect, there exist a few of formats that allow this.

\subsection{Cavities}
Cavities are a bit harder to create as the idea of it is that in order to create a Polyglot file we use padded, unused zero-filled memory
within the first file. Usually this is the case for executables, such as ELF, Mach-O or ROMs or ISOs. It can happen that there exists enough
zero-filled static memory that could fit a whole another file and again any offsets pointers or reference within the second file would need to be
adjusted \cite{Ange2}.

\subsection{Zippers}
Zippers are the more complicated way of creating Polyglots out of the other options. The idea is that the two files are sliced up into
pieces and merged together by making use of the Metadata or Comment blocks of both file formats. One of the files must have the requirment
that the header can start later than at offset 0. In both files the references pointers and offsets would need to be adjusted \cite{Ange2}.

\subsection{Parasites}
Parasites are very similar to Cavities, but instead of using zero-filled unused memory we abuse the spec of the file format by hiding
the second file within the first file, by using Metadata or Comments tags that are allowed by the specifications. These block or tags aren't displayed
to the user when the file is being read. Usually they are limited in size which varries across the file formats.
