\section{Conclusion}

I've tried creating multiple Polyglot files that still work today with the latest software. There are some catches though.
Adobe Acrobat Reader (which directly follows the spec of PDF) doesn't consider the PDF polyglots as valid anymore and has been
improved over the last decade to disallow opening polyglot files within it. However other PDF viewers such as Preview on MacOS or
Document Viewer on linux have no problem with these polyglots whatsoever including browsers such as Safari or Firefox. The same is true for mobile devices 
such as Iphone or Remarkable Tablet.

Creating Polyglots files that work on MacOs is a bit more harder to do. Especially if you want to create a polyglot with a binary of some sort
as MacOs requires the binaries to be signed and its a bit harder to distribute polyglots with self-signed certificates, it's doabble just harder.
Also MacOS has some protection for Zip files, it requires that nothing starts before the ZIP file itself, so if you just try to insert a single byte
it will just straight reject it, however if you use the command-line to unzip a polyglot is still works.

Further, Google Chrome has also protection agains opening polyglots that contain some sort of binary data. I've tried this and chrome just rejects them.

I haven't encountered any issues with polyglots on Linux. I'm not sure if they'll work on windows as I haven't targeted windows as the primary OS, though I believe
some of them if not most would also work.
